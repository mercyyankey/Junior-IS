% This is a template for your written document.
%
% To compile using latexmk on the command line, run the following: 
% latexmk -pdf main.tex

\documentclass[12pt]{article}
\usepackage{setspace}
\usepackage{graphicx} % used for includegraphics
\singlespace
\usepackage[left=1in,right=1in,top=1in,bottom=1in]{geometry}

\title{\textbf{Bridging Understanding to Speaking: A Language Learning Application for African Heritage Speakers}}
\author{Mercy Yankey}

\begin{document}

\maketitle

Heritage speakers of under-researched African languages often demonstrate strong listening comprehension while struggling with spontaneous language production. Many individuals raised in multilingual households can understand conversations fluently but lack confidence, vocabulary access, or grammatical control when speaking. This gap is particularly evident among heritage speakers of languages such as Twi, where formal instructional resources are limited and language learning is often informal, inconsistent, and orally transmitted.

This project proposes the design of a language-learning application tailored to African heritage speakers, beginning with Twi. The goal of the application is to support the transition from comprehension to active language production by combining structured lessons, gamified practice, narrative-based learning, and low-effort review tools. By offering multiple learning pathways, the system aims to accommodate diverse cognitive styles, attention levels, and learning preferences while addressing the specific challenges faced by heritage language learners.

\section*{Project Significance}
Despite this targeted approach, many African languages spoken by heritage communities remain significantly under-researched and under-supported in contemporary language learning technologies. While widely spoken global languages benefit from extensive digital resources, standardized curricula, and high-quality translation systems, languages such as Twi are often excluded from these developments. As a result, heritage speakers frequently rely on informal transmission within families and communities, which can lead to strong comprehension but limited speaking confidence and inconsistent language use.

Most language-learning applications target second-language learners rather than heritage speakers. These platforms assume users lack prior exposure to the language and require a linear lesson progression, which fails to reflect the lived experiences of heritage learners, who already hold partial linguistic knowledge. Furthermore, mainstream translation tools and digital dictionaries frequently provide inaccurate or incomplete representations of African languages, creating barriers to effective learning and discouraging sustained engagement.

This project is significant because it centers on the needs of African heritage speakers and treats their linguistic background as an asset rather than an insufficiency. By designing a system that supports multiple learning pathways, including structured lessons, interactive games, narrative-based contexts, and low-effort review modes, the proposed application aims to bridge the gap between understanding and speaking. In doing so, this project contributes to broader conversations around equitable access to educational technology, culturally responsive design, and the preservation and revitalization of under-supported languages through thoughtful computational systems.

\begin{figure}[h]
\begin{center}
\includegraphics[scale=0.7]{methodology.png}
\caption{Archie}
\label{fig:method}       % Give a unique label
\end{center}
\end{figure}


\newpage
\section*{Appendix}
A concise list of features / user stories in the order in which they will be built. A few examples are below to demonstrate the expected scope and level of granularity; you will have more features than this.
\begin{itemize}
	\item Default picture display on web application.
	\item On a button-click, user can separate the image into foreground and background.
	\item User can select a picture from their desktop.
	\item Selected picture displays on the web application.
\end{itemize}


\bibliographystyle{acm}
\bibliography{bibliography.bib}

\end{document}
